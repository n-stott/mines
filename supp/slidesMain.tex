\documentclass{beamer}
\usetheme{Madrid}

\usepackage[utf8]{inputenc}
\usepackage{amsfonts}
\usepackage{amsmath}
\usepackage{amssymb}
\usepackage{mathrsfs}
\usepackage[french]{babel}

%\usepackage{arev}

\usepackage[babel=true]{csquotes}

\usepackage{listings}

\usepackage[center]{caption}
\usepackage{tabularx}


\usepackage{tikz}
\usetikzlibrary{calc, trees, positioning, arrows, shapes, shapes.multipart, shadows, matrix, decorations.pathreplacing, decorations.pathmorphing}

\lstset{escapeinside={<@}{@>}}


\title{Introduction à OpenGL avec GLUT}
%\author[Stott, Bouchiba]{Nikolas Stott \inst{1} \and Hassan Bouchiba \inst{2}}
%\institute[INRIA, MINES]{\inst{1} INRIA Saclay \and \inst{2} MINES ParisTech}
\author[Nikolas Stott]{Nikolas Stott\inst{1} \and Hassan Bouchiba \inst{2}}
\institute[LocalSolver]{\inst{1} LocalSolver \and \inst{2} Terra3d}
\titlegraphic{\includegraphics[width=0.3\textwidth]{img/opengl}}
\date{\today}

\AtBeginSection[]
{
\begin{frame}<beamer>{Plan}
\tableofcontents[currentsection]
\end{frame}
}

\newif\iffull
%\fulltrue

\begin{document}

\begin{frame}
\titlepage
\end{frame}

\begin{frame}{Plan}
\tableofcontents
\end{frame}


\section{OpenGL et GLUT : présentation}
\begin{frame}{Qui fait quoi ?}

 \begin{block}{GLUT : haut niveau}
 Interface de programmation :
 \begin{itemize} 
 \item Fenêtrage, 
 \item Visualisation,
 \item Input-output utilisateur
 \end{itemize}
 \end{block}
 
 \begin{block}{OpenGL : bas niveau}
 Interface avec le hardware graphique :
 \begin{itemize}
 \item Commandes de création de géométrie, de couleurs
 \item Manipulation des objets
 \end{itemize}
 \end{block}

\end{frame}

\begin{frame}{Qu'est ce qu'OpenGL sait faire ?}
	\begin{columns}
		\begin{column}{0.55\textwidth}
			\begin{block}{Modélisation/Visualisation}
				\begin{itemize}
					\item Création de géométries complexes
					\item Habillage de la géométrie : couleur, texture, éclairage...
				\end{itemize}
			\end{block}
		\end{column}
		\begin{column}{0.4\textwidth}
			\centering
			\includegraphics[width=0.8\textwidth]{img/model_earth}
		\end{column}
	\end{columns}
	~\\
	\pause
	\begin{columns}
		\begin{column}{0.4\textwidth}
			\centering
			\includegraphics[width=0.8\textwidth]{img/animation}
		\end{column}
		\begin{column}{0.55\textwidth}
			\begin{block}{Animation}
				OpenGL permet d'animer :
				\begin{itemize}
					\item la caméra dans la scène
					\item les objets dans la scène
					\item le maillage des objets
				\end{itemize}
			\end{block}
		\end{column}
	\end{columns}
\end{frame}


%\iffalse

\section{\'El\'ements de modélisation avec OpenGL}

\begin{frame}{La brique de base : le sommet}
	\begin{alertblock}{Dans le cadre de ce cours, ...}
		... l'ordinateur ne conna\^it que le point (vecteur ou \emph{vertex}) !
	\end{alertblock}
	\begin{itemize}
		\item[$\Rightarrow$] Pas de courbe, pas de boule, pas de géométrie lisse.
	\end{itemize}
	\pause
	\begin{block}{Assemblage de géométrie}
		On peut créer des objets simples à partir de sommets:
		\begin{itemize}
			\item lignes (ou \emph{edges}),
			\item polygones (ou \emph{faces}).
		\end{itemize}
	\end{block}
	\begin{itemize}
		\item[$\Rightarrow$] On va assembler ces objets simples en objets plus complexes...
	\end{itemize}
\end{frame}

\begin{frame}{C'est tout ?}
	\begin{block}{Ce n'est pas suffisant !}
		Il faut également déclarer des informations supplémentaires sur l'objet:
		\begin{columns}
			\begin{column}{0.5\textwidth}
				\begin{itemize}
					\item Orientation des faces,
					\item Couleur des faces,
				\end{itemize}
			\end{column}
			\begin{column}{0.5\textwidth}
				\begin{itemize}
					\item Matériaux de l'objet,
					\item Texture des faces...
				\end{itemize}			
			\end{column}
		\end{columns}
		~\\
		pour qu'il interagisse avec son environnement: lumières...
	\end{block}
	\begin{columns}
		\begin{column}{0.5\textwidth}
			\centering
			\begin{figure}
				\includegraphics[width=0.6\textwidth]{img/no_shade_sphere}
				\caption*{Sans aucune information:}
			\end{figure}
		\end{column}
		\begin{column}{0.5\textwidth}
			\centering
			\begin{figure}
				\includegraphics[width=0.6\textwidth]{img/shaded_sphere}
				\caption*{Avec orientation et matériaux:}
			\end{figure}
		\end{column}
	\end{columns}
\end{frame}

\begin{frame}{Pas votre lampe habituelle}
	\begin{block}{Les composantes lumineuses du modèle de Phong}
		Une lumière OpenGL est caractérisée par 5 paramètres:
		\begin{itemize}
			\item Position : vecteur XYZ(W)
			\item Direction du spot : vecteur direction XYZ
			\item Intensité Diffuse : vecteur RGBA
			\item Intensité Spéculaire : vecteur RGBA
			\item Intensité Ambiante : vecteur RGBA
		\end{itemize}
	\end{block}
	\begin{figure}
		\includegraphics[width=\textwidth]{img/phong}
	\end{figure}
\end{frame}

\begin{frame}{Lumières et Normales}
	\begin{block}{Influence de la normale}
		Sur une face, l'éclairage est calculé avec:
		$
		\left\{
		\begin{tabular}{p{4cm}}
		\text{le vecteur lumineux incident,} \\
		\text{la normale \`a la face}		
		\end{tabular}
		\right.$.
	\end{block}
	\vspace{2mm}
	\begin{columns}
		\begin{column}{0.5\textwidth}
			\centering
			\includegraphics[width=0.8\textwidth]{img/diffuseAngle}
		\end{column}
		\begin{column}{0.5\textwidth}
			\centering
			\includegraphics[width=\textwidth]{img/specular}
		\end{column}
	\end{columns}
	\begin{alertblock}{}
		Il faut donner les normales à l'ordinateur: il ne sait pas les calculer...
	\end{alertblock}
\end{frame}

\begin{frame}{Couleur et Textures}
	\begin{columns}
		\begin{column}{0.65\textwidth}
			\begin{block}{Colorier la géométrie}
				\begin{itemize}
					\item Ce sont les sommets qui portent l'information de couleur;
					\item Les faces sont coloriées par interpolation.
				\end{itemize}
			\end{block}
		\end{column}
		\begin{column}{0.3\textwidth}
			\centering
			\includegraphics[width=0.6\textwidth]{img/triang}
		\end{column}
	\end{columns}
	\pause
	\begin{block}{Texturer la géométrie}
		\begin{itemize}
			\item On attribue un point de la texture à chaque sommet ;
			\item Les faces sont texturées par interpolation.
		\end{itemize}
	\end{block}
	\begin{columns}
		\begin{column}{0.4\textwidth}
			\centering
			\includegraphics[width=0.7\textwidth]{img/model_earth}
		\end{column}
		\begin{column}{0.6\textwidth}
			\centering
			\includegraphics[width=1\textwidth]{img/uv_map}
		\end{column}
	\end{columns}
\end{frame}

\begin{frame}{Ce que l'on ne va pas apprendre à faire}
	\begin{alertblock}{Les ombres}
		Les ombres portées ne sont pas calculées automatiquement : c'est à l'utilisateur de les calculer.
	\end{alertblock}
	\begin{alertblock}{Texturer un objet}
		C'est encore à l'utilisateur de définir comment la texture s'applique sur l'objet (UV mapping).
	\end{alertblock}
	\begin{alertblock}{Traitement des shaders}
		Pas de traitement en détail des normales, effets post-traitement, etc... \\
		Pas de canaux de textures améliorés : normal map, bump map, displacement map, ...
	\end{alertblock}
\end{frame}


\section{Les commandes OpenGL}

\subsection{Les commandes basiques}

\begin{frame}
\frametitle{Les briques élémentaires (1)}
	\begin{block}{Les sommets}
		Les sommets sont déclarés par la commande \verb!glVertex3f(x,y,z)!.\\
		Exemples :
		\begin{itemize}
			\item \alert{\verb!glVertex3f(2.4f,4.1f,-0.1f);!}
		\end{itemize}
	\end{block}
%	\begin{columns}
%		\begin{column}{0.7\textwidth}
%			\centering
%			\begin{block}{Par défaut}
%				\begin{itemize}
%					\item \verb!glVertex2?! place les points dans le plan $z = 0$ (et $w = 1$)
%					\item \verb!glVertex3?! fixe $w$ à $1$.
%				\end{itemize}
%			\end{block}
%		\end{column}
%		\begin{column}{0.3\textwidth}
%		\centering
%			\includegraphics[width=0.7\textwidth]{img/coords}
%		\end{column}
%	\end{columns}
\end{frame}

\begin{frame}[fragile]
\frametitle{Les briques élémentaires (2)}
	\begin{block}{La déclaration de primitive}
		L'assemblage de sommets en faces se fait entre les instructions\\
		\begin{figure}[h]
			\centering
			\begin{tabular}{c}
				\begin{lstlisting}
glBegin(TYPE_DE_LA_PRIMITIVE);
  //declarations optionnelles
  glVertex3f(...);
  ...
  glVertex3f(...);
  //declarations optionnelles
glEnd();
				\end{lstlisting}
			\end{tabular}
		\end{figure}
		Nous verrons les autres déclarations plus loin.
	\end{block}
%	\begin{exampleblock}{}
%		Attention : puisque nous travaillons sur de la géométrie, OpenGL doit être configuré en mode \verb!GL_MODELVIEW! : \verb!glMatrixMode(GL_MODELVIEW);! !
%	\end{exampleblock}
\end{frame}

\begin{frame}
\frametitle{Primitives géométriques disponibles}
\begin{columns}
	\begin{column}{0.3\textwidth}
		\centering		
		\includegraphics[width=\textwidth]{img/GL_POINTS}
		\\ { GL\_POINTS}
	\end{column}
	\begin{column}{0.3\textwidth}
		\centering	
		\includegraphics[width=\textwidth]{img/GL_LINES}
		\\ { GL\_LINES}
	\end{column}
%	\begin{column}{0.2\textwidth}
%		\centering	
%		\includegraphics[width=\textwidth]{img/GL_LINE_STRIP}
%		\\ {\tiny GL\_LINE\_STRIP}
%	\end{column}
	\begin{column}{0.3\textwidth}
		\centering	
		\includegraphics[width=\textwidth]{img/GL_LINE_LOOP}
		\\ { GL\_LINE\_LOOP}
	\end{column}
\end{columns}
~\\
~\\
\pause
\begin{columns}
	\begin{column}{0.3\textwidth}
		\centering		
		\includegraphics[width=\textwidth]{img/GL_TRIANGLES}
		\\ { GL\_TRIANGLES}
	\end{column}
%	\begin{column}{0.2\textwidth}
%		\centering	
%		\includegraphics[width=\textwidth]{img/GL_TRIANGLE_STRIP}
%		\\ {\tiny GL\_TRIANGLE\_STRIP}
%	\end{column}
%	\begin{column}{0.3\textwidth}
%		\centering	
%		\includegraphics[width=\textwidth]{img/GL_TRIANGLE_FAN}
%		\\ { GL\_TRIANGLE\_FAN}
%	\end{column}
	\begin{column}{0.3\textwidth}
		\centering		
		\includegraphics[width=\textwidth]{img/GL_QUADS}
		\\ { GL\_QUADS}
	\end{column}
%	\begin{column}{0.2\textwidth}
%		\centering	
%		\includegraphics[width=\textwidth]{img/GL_QUAD_STRIP}
%		\\ {\tiny GL\_QUAD\_STRIP}
%	\end{column}
%	\begin{column}{0.2\textwidth}
%		\centering	
%		\includegraphics[width=\textwidth]{img/GL_POLYGON}
%		\\ {\tiny GL\_POLYGON}
%	\end{column}
\end{columns}
\end{frame}

\begin{frame}[fragile]
\frametitle{Exemples simples de modélisation : cylindre}
	\vspace{-5mm}
	\begin{columns}[T]
		\begin{column}{0.45\textwidth}
			\begin{block}{Cylindre}
			\centering
				\includegraphics[width=0.6\textwidth]{img/cone_base}
				~\\
				\includegraphics[width=0.6\textwidth]{img/cylindre}
			\end{block}
		\end{column}
		\pause
		\begin{column}{0.45\textwidth}
			\begin{exampleblock}{Réponse}
				\begin{figure}[h]
					\begin{tabular}{c}
						\centering
						\begin{lstlisting}[basicstyle=\small]

for(int i=0; i<p; ++i){
  glBegin(GL_TRIANGLES);
  ... //base inferieure
  glEnd();
  glBegin(GL_TRIANGLES);
  ... //base superieure
  glEnd();
  glBegin(GL_QUADS);
  ... //faces laterales
  glEnd();
}
						\end{lstlisting}
					\end{tabular}
				\end{figure}
			\end{exampleblock}
			\pause
			\begin{exampleblock}{Une autre solution}
				\begin{figure}[h]
					\centering
					\begin{tabular}{c}
						\begin{lstlisting}[basicstyle=\small]
gluCylinder(...);
						\end{lstlisting}
					\end{tabular}
				\end{figure}
			\end{exampleblock}
		\end{column}
	\end{columns}
\end{frame}



\begin{frame}[fragile]
\frametitle{Exemple de modélisation : cône}
	\begin{columns}[T]
		\begin{column}{0.45\textwidth}
			\begin{block}{Cône}
				\centering
				\includegraphics[width=0.75\textwidth]{img/cone}
			\end{block}
		\end{column}
		\pause
		\begin{column}{0.45\textwidth}
			\begin{exampleblock}{Réponse}
				\begin{figure}[h]
					\centering
					\begin{tabular}{c}
						\begin{lstlisting}[basicstyle=\small]

for(int i=0; i<p; ++i){
  glBegin(GL_TRIANGLES);
  ... //face inferieure
  glEnd();
  glBegin(GL_TRIANGLES);
  ... //faces laterales
  glEnd();
}
						\end{lstlisting}
					\end{tabular}
				\end{figure}	
			\end{exampleblock}	
		\end{column}
	\end{columns}
	\pause
	\begin{exampleblock}{Une autre solution}
		\begin{figure}[h]
			\centering
			\begin{tabular}{c}
				\begin{lstlisting}[basicstyle=\small]
glutSolidCone(...);
				\end{lstlisting}
			\end{tabular}
		\end{figure}
	\end{exampleblock}
\end{frame}


\subsection{Commandes avancées}

\begin{frame}{OpenGL est une machine à états}
\begin{block}{Deux nouvelles commandes}
\begin{itemize}
\item glNormal3f
\item glColor3f
\end{itemize}
\end{block}
\begin{block}{Machine à états}
Executer \verb!glColor3f(1,1,1)! choisit la "couleur courante". Tout sera dessiné avec cette couleur jusqu'à ce que la couleur soit changée par une nouvelle exécution de \verb!glColor()!.
\end{block}
\end{frame}

\begin{frame}[fragile]
\frametitle{Commandes à disposition entre \verb!glBegin()! et \verb!glEnd()! (1)}
	\begin{block}{\verb!glVertex3f(x,y,z)!}
	\end{block}
	\begin{block}{\verb!glNormal3f(x,y,z)!}
		Déclarer la normale au polygone que l'on trace.\\
		\alert{Une normale déclarée est utilisée jusqu'à être remplacée}
%		\alert{Utile} : initialiser \verb!glEnable(GL_NORMALIZE);! : normalisation auto.
		\begin{minipage}{\linewidth}
			\begin{columns}[T]
				\begin{column}{0.45\textwidth}
					\begin{exampleblock}{}
						\begin{lstlisting}[basicstyle=\small]
glBegin(GL_QUADS);
  <@\textcolor{red}{glNormal3f(0,1,0);}@>
    glVertex3f(1,0,1);
    glVertex3f(1,0,-1);
    glVertex3f(-1,0,-1);
    glVertex3f(-1,0,1);
glEnd();
						\end{lstlisting}
					\end{exampleblock}
				\end{column}
				\begin{column}{0.45\textwidth}
					\begin{exampleblock}{}
						\begin{lstlisting}[basicstyle=\small]
glBegin(GL_TRIANGLES);
  <@\textcolor{red}{glNormal3f(0,1,0);}@>
    glVertex3f(1,0,1);
    glVertex3f(1,0,-1);
    glVertex3f(-1,0,-1);
  <@\textcolor{red}{glNormal3f(0,1,0);}@>
    glVertex3f(1,0,1);
    glVertex3f(-1,0,-1);
    glVertex3f(-1,0,1);
glEnd();
						\end{lstlisting}
					\end{exampleblock}
				\end{column}
			\end{columns}
		\end{minipage}
	\end{block}
\end{frame}

\begin{frame}[fragile]
\frametitle{Commandes à disposition entre \verb!glBegin()! et \verb!glEnd()! (2)}
	\begin{block}{\verb!glColor3f(r,g,b)!}
		\begin{columns}
			\begin{column}{0.8\textwidth}
				Déclarer la couleur à utiliser dans la suite du programme.\\
				\verb!r!,\verb!g!,\verb!b!,\verb!a! doivent prendre des valeurs entre $0.0$ et $1.0$.\\
				Par défaut : \verb!glColor4d(1,1,1,1);!.\\
				Pour activer la composante alpha : \verb!glEnable(GL_BLEND);!.
			\end{column}
			\begin{column}{0.15\textwidth}
				\includegraphics[width=\textwidth]{img/triang}
			\end{column}
		\end{columns}
		\begin{figure}
			\centering
			\begin{tabular}{c}
				\begin{lstlisting}[basicstyle=\small]
glBegin(GL_TRIANGLE);
  glNormal3f(0, 0, 1);
    <@\textcolor{red}{glColor3f(1, 0, 0);}@>
      glVertex3f(,,);
    <@\textcolor{red}{glColor3f(1, 1, 0);}@>
      glVertex3f(,,);
    <@\textcolor{red}{glColor3f(1, 1, 1);}@>
      glVertex3f(,,);
glEnd();
				\end{lstlisting}
			\end{tabular}
		\end{figure}
	\end{block}
\end{frame}

%\begin{frame}[fragile]
%\frametitle{Commandes à disposition entre \verb!glBegin()! et \verb!glEnd()! (3)}
%	\begin{columns}
%		\begin{column}{0.6\textwidth}
%			\begin{block}{\verb!glMaterial?(face, gl\_param, values)!}
%				\verb!face! détermine quelle face est calculée :
%				\begin{itemize}
%					\item \alert{ \verb!GL\_FRONT\_AND\_BACK! }
%				\end{itemize}
%				gl\_param est le paramètre du matériau :
%				\begin{itemize}
%					\item \verb!GL_AMBIENT! : 4 variables $[-1.0,1.0]$
%					\item \verb!GL_DIFFUSE! : 4 variables $[-1.0,1.0]$
%					\item \verb!GL_SPECULAR! : 4 variables $[-1.0,1.0]$
%					\item \verb!GL_EMISSION! : 4 variables $[-1.0,1.0]$
%					\item \verb!GL_SHININESS! : 1 variable $[0,128]$
%				\end{itemize}
%				\verb!values! est un pointeur vers le tableau de paramètres
%			\end{block}
%		\end{column}
%		\begin{column}{0.35\textwidth}
%			\includegraphics[width=\textwidth]{img/light_material}
%		\end{column}
%	\end{columns}
%\end{frame}

\begin{frame}{Récupération du TP}

\begin{block}{Dans un terminal:}
\begin{itemize}
\item git clone https://github.com/n-stott/mines.git
\end{itemize}
\end{block}

\begin{block}{Architecture}
\begin{itemize}
\item Boids:
\begin{itemize}
\item  le gros TP (plus tard)
\end{itemize}
\item Canvas:
\begin{itemize}
\item exercice de cours (maintenant)
\end{itemize}
\item Supp:
\begin{itemize}
\item supports de cours
\end{itemize}
\end{itemize}
\end{block}

\begin{block}{Compilation}
Dans un terminal, dans le dossier de l'exercice/du TP, exécuter:
\begin{itemize}
\item sh build.sh
\end{itemize}
\end{block}

\end{frame}





\subsection{Transformations géométriques}

\begin{frame}
\frametitle{Les transformations élémentaires}
	\begin{alertblock}{Principe de transformation}
		Chaque commande de transformation transforme le repère de dessin local.
	\end{alertblock}
	\begin{block}{Translation}
		\verb!glTranslatef(x,y,z)! 
	\end{block}
	\begin{block}{Changement d'échelle}
		\verb!glScalef(x,y,z)! multiplie l'échelle par \verb!x!,\verb!y!,\verb!z! dans les directions $x$,$y$,$z$ respectivement.
	\end{block}
	\begin{block}{Rotation autour d'un axe}
		\verb!glRotatef(t,x,y,z)! effectue une rotation de $t$ (en degrés) autour de l'axe $[x\;y\;z]^T$
	\end{block}
\end{frame}

\begin{frame}[fragile]
\frametitle{Modélisation, transformation et animation}
\setbeamercovered{invisible}
	\begin{exampleblock}{Exercice}
		Comment faire pour animer un objet dans une scène ? Par exemple, faire tourner un carré autour d'une de ses diagonales ? Et un cube ?
	\end{exampleblock}
	\pause
	\begin{columns}[T]
		\begin{column}{0.58\textwidth}
			\begin{block}{Solution 1 :}
			%	Animer la position des sommets lors du tracé.
			%	\vspace{-5mm}
				\begin{figure}[h]
				\centering
				\begin{tabular}{c}
				\begin{lstlisting}[basicstyle=\small]
glBegin(GL_QUADS);
  ...
  glVertex3f(cosf(t),sinf(t),0);
  ...
glEnd();
				\end{lstlisting}
				\end{tabular}
				\end{figure}
			\end{block}
		\end{column}
		\begin{column}{0.40\textwidth}
			\begin{block}{Solution 2:}
			%	Tracer un cube et effectuer la transformation nécessaire.
				\begin{figure}[h]
				\centering
				\begin{tabular}{c}
				\begin{lstlisting}[basicstyle=\small]
glRotatef(t,0,0,1);
glBegin(GL_QUADS);
  ...
  glVertex3f(1,0,0);
  ...
glEnd();
glRotatef(-t,0,0,1);
				\end{lstlisting}
				\end{tabular}
				\end{figure}
			\end{block}
		\end{column}
	\end{columns}
	\begin{block}{}
		En général, la 2ème solution est préférable, surtout si les objets sont complexes : calcul sur le GPU et non sur le CPU.
	\end{block}
\end{frame}

\begin{frame}[fragile]
\frametitle{Manipulation de transformations (1)}
	\begin{exampleblock}{Pas très pratique...}
		\begin{columns}
			\begin{column}{0.45\textwidth}
				\begin{block}{}
					Faire toutes les transformations en double, avant et après avoir tracé mon objet ?
				\end{block}
			\end{column}
			\begin{column}{0.45\textwidth}
				\begin{block}{}
					\begin{figure}[h]
						\centering
						\begin{tabular}{c}
							\begin{lstlisting}
glRotatef(t,0,0,1);
...
glRotatef(-t,0,0,1);
							\end{lstlisting}
						\end{tabular}
					\end{figure}
				\end{block}
			\end{column}
		\end{columns}
	\end{exampleblock}
	\pause
	\begin{block}{Pile de matrices : sauvegarder un état de transformation}
		On choisit la pile de matrices sur laquelle on agit avec \verb!glMatrixMode( GL_MODELVIEW | GL_PROJECTION )!.
		\begin{itemize}
			\item \verb!glLoadIdentity()! : annuler toutes les transformations
			\item \verb!glPushMatrix()! : sauvegarder l'état de transformation courant
			\item \verb!glPopMatrix()! : retour à l'état précédemment enregistré
		\end{itemize}
	\end{block}
\end{frame}

\begin{frame}[fragile]
\frametitle{Manipulation de transformations (2)}
\begin{exampleblock}{Beaucoup plus pratique}
On reprend l'exemple précédent :
\begin{columns}
	\begin{column}{0.45\textwidth}
		\begin{block}{}
			\begin{figure}[h]
			\centering
			\begin{tabular}{c}
			\begin{lstlisting}
glRotatef(t,0,0,1);
glTranslatef(4,3,1);
...
glTranslatef(-4,-3,-1);
glRotatef(-t,0,0,1);
			\end{lstlisting}
			\end{tabular}
			\end{figure}
		\end{block}
	\end{column}
	\begin{column}{0.45\textwidth}
		\begin{block}{}
			\begin{figure}[h]
			\centering
			\begin{tabular}{c}
			\begin{lstlisting}
glPushMatrix();
  glRotatef(t,0,0,1);
  glTranslatef(4,3,1);
  ...
glPopMatrix();
			\end{lstlisting}
			\end{tabular}
			\end{figure}
		\end{block}
	\end{column}
\end{columns}
\end{exampleblock}
\end{frame}

%\subsection{L'éclairage}
%
%\begin{frame}[fragile]
%\frametitle{L'éclairage dans OpenGL}
%\begin{columns}
%\begin{column}{0.5\textwidth}
%\begin{block}{Type d'ombrage :}
%La commande  \verb!glShadeModel(??)! permet de choisir entre 2 types d'ombrages :
%\begin{itemize}
%	\item \verb!GL_FLAT!
%	\item \verb!GL_SMOOTH!
%\end{itemize}
%\centering
%\includegraphics[width=0.8\textwidth]{img/shademodel}
%\end{block}
%\end{column}
%%\begin{column}{0.45\textwidth}
%%\includegraphics[width=\textwidth]{img/onedoesnotsimply}
%%\end{column}
%\end{columns}
%\end{frame}
%
%\begin{frame}
%\frametitle{Le modèle d'éclairage dans OpenGL}
%
%\begin{columns}
%\begin{column}{0.45\textwidth}
%\begin{block}{Composantes lumineuses}
%\begin{itemize}
%	\item Composante \'Emissive
%	\item Composante Ambiante
%	\item Composante Diffuse
%	\item Composante Spéculaire
%\end{itemize}
%\end{block}
%\end{column}
%\begin{column}{0.45\textwidth}
%\begin{block}{Types de sources lumineuses}
%\begin{itemize}
%	\item Source Ambiante
%	\item Source Ponctuelle
%	\item Source Directionnelle
%	\item Source Spot
%\end{itemize}
%\end{block}
%\end{column}
%\end{columns}
%~\\
%~\\
%\begin{columns}
%\begin{column}{0.45\textwidth}
%	\centering
%	\includegraphics[width=0.7\textwidth]{img/light_material}
%\end{column}
%\begin{column}{0.45\textwidth}
%	\centering
%	\includegraphics[width=0.7\textwidth]{img/lighttypes}
%\end{column}
%\end{columns}
%\end{frame}
%
%\begin{frame}[fragile]
%\frametitle{Utiliser les lumières}
%\begin{block}{Les lumières dans OpenGL}
%\begin{itemize}
%	\item OpenGL sait gérer jusqu'à 8 sources de lumières simultanément,
%	\item Elles sont nommées \verb!GL_LIGHT!$i$, avec $0 \leq i < 8$.
%\end{itemize}
%\end{block}
%\begin{block}{Activation}
%\begin{itemize}
%	\item{ Configuration d'OpenGL pour utiliser les lumières :
%	\begin{center}
%		\verb!glEnable(GL_LIGHTING);!\\
%	\end{center}
%	}
%	\item{ Activation d'une lumière :
%	\begin{center}
%		\verb!glEnable(GL_LIGHT0);!
%	\end{center}
%	}
%	\item{ On déclare les couleurs de faces comme des matériaux :
%		\verb!glEnable(GL_COLOR_MATERIAL);!
%	}
%\end{itemize}
%\end{block}
%\end{frame}
%
%\begin{frame}[fragile]
%\frametitle{Les commandes de lumière}
%\begin{block}{\verb!glLight(gl\_light,gl\_param,values)!}
%Les paramètres ajustables et leurs valeurs :
%\begin{itemize}
%	\item \verb!GL_AMBIENT! : 4 variables (intensité RGBA)
%	\item \verb!GL_DIFFUSE! : 4 variables (intensité RGBA)
%	\item \verb!GL_SPECULAR! : 4 variables (intensité RGBA)
%	\item \verb!GL_POSITION! : 4 variables (position XYZW)
%	\item \verb!GL_SPOT_DIRECTION! : 3 variables (vecteur direction)
%\end{itemize}
%\end{block}
%\begin{block}{\verb!glLightModel(gl\_param,gl\_value)! (Description)}
%Offrir plus de contrôle sur la compréhension des paramètres lumineux :
%\begin{itemize}
%	\item \verb!GL_LIGHT_MODEL_AMBIENT! : contrôle l'intensité de la lumière ambiante
%	\item \verb!GL_LIGHT_MODEL_COLOR_CONTROL! : contrôle des éventuels conflits entre textures et lumière
%\end{itemize}
%\end{block}
%\end{frame}

\subsection{Gestion de la caméra}

\begin{frame}
\frametitle{Utilisation simple de la caméra}
\begin{block}{Une fonction GLU utile}
	\begin{itemize}
		\item{\verb!gluLookAt(eyeX,eyeY,eyeZ,atX,atY,atZ,upX,upY,upZ)!
%			\begin{itemize}
%				\item \verb!fovy! : ouverture verticale en degrés
%				\item \verb!aspect! : ratio largeur/hauteur de l'image
%				\item \verb!zNear! : plan de coupure proche
%				\item \verb!zFar! : plan de couûre lointain
%			\end{itemize}
		}
	\end{itemize}
	\centering
	\includegraphics[width=0.5\textwidth]{img/ortho}\\
	On manipule des matrices géométriques :\\
	\verb!glMatrixMode(GL\_MODELVIEW);!
\end{block}
\end{frame}


\section{Les fonctions principales GLUT}

%\begin{frame}{}
%
%\begin{center}
%Présentation du TP
%\end{center}
%
%\end{frame}

%\begin{frame}[fragile]
%\frametitle{La fonction Main}
%\begin{block}{Contenu}
%La fonction main doit :
%\begin{itemize}
%	\item initialiser GLUT : \verb!glutInit(&argc,argv);!
%	\item paramétrer l'affichage avec \verb!glutInitDisplayMode( ... );! :\\ \verb!GLUT_RGBA!, \verb!GLUT_DEPTH!, \verb!GLUT_SINGLE! ou \verb!GLUT_DOUBLE!
%	\item créer la fenêtre : \verb!glutCreateWindow("C'est bientôt fini ;)");!
%	\item initialiser les variables/objets du programme
%	\item {déclarer les fonctions
%		\begin{itemize}
%			\item de dessin : \verb!glutDisplayFunc( ... );!
%			\item de redimensionnement : \verb!glutReshapeFunc( ... );!
%			\item d'interaction souris : \verb!glutMouseFunc( ... );!
%			\item d'interaction clavier : \verb!glutKeyboardFunc( ... );!
%			\item d'évolution autonome : \verb!glutIdleFunc( ... );!
%			\item de temporisation : \verb!glutTimerFunc( ... );!
%		\end{itemize}
%	}
%	\item lancer la boucle infinie : \verb!glutMainLoop();!
%\end{itemize}
%\end{block}
%\end{frame}
%
%\begin{frame}{Autres fonctions (1)}
%\begin{block}{Fonction de dessin}
%Donnée en paramètre de \verb!glutDisplayFunc(...)!\\
%Elle ne prend rien en paramètre.\\
%Son rôle est de tracer l'image courante :
%\begin{itemize}
%	\item effacer l'image précédente : \verb!glClear(GL\_COLOR\_BUFFER\_BIT);!
%	\item dessiner ce que l'utilisateur souhaite
%	\item demander de l'afficher : \verb!glFlush();! ou \verb!glSwapBuffers();!
%\end{itemize}
%\end{block}
%\pause
%\begin{block}{Fonction de redimensionnement}
%Donnée en paramètre de \verb!glutReshapeFunc( ... );!\\
%Elle prend en paramètre les dimensions du viewport.\\
%Elle doit assurer la cohérence de la fenêtre de tracé :
%\begin{itemize}
%	\item déclarer le viewport : \verb!glViewport(x1,y1,x2,y2);!
%	\item charger les paramètres caméra initiaux
%\end{itemize}
%\end{block}
%\end{frame}
%
%\begin{frame}{Autres fonctions (2)}
%\begin{block}{Fonction d'interaction souris}
%Donnée en paramètre de \verb!glutMouseFunc( ... );!\\
%Elle prend en paramètre le bouton activé, l'état du bouton et la position écran lors de l'action.
%\begin{itemize}
%	\item Le bouton prend les valeurs \verb!GLUT\_LEFT/MIDDLE/RIGHT\_BUTTON!
%	\item L'état prend les valeurs \verb!GLUT\_UP! et \verb!GLUT\_DOWN!
%\end{itemize}
%\end{block}
%\pause
%\begin{block}{Fonction d'interaction clavier}
%Donnée en paramètre de \verb!glutKeyboardFunc( ... );!\\
%Elle prend en paramètre la touche activée et la position écran de la souris lors de l'action.
%\end{block}
%\end{frame}
%
%\begin{frame}{Autres fonctions (3)}
%\begin{block}{Fonction d'évolution autonome}
%Donnée en paramètre de \verb!glutIdleFunc( ... );!\\
%Appelée lorsqu'aucune action n'est déclenchée, elle ne prend aucun paramètre. C'est la fonction qui calcule le nouvel état du système, etc.
%\end{block}
%\pause
%\begin{block}{Fonction de temporisation}
%Donnée en paramètre de \verb!glutTimerFunc( ... );!\\
%Fonction avancée qui permet d'introduire des paramètres temporels dans le programme.\\
%\verb!glutTimerFunc(DeltaT,timer,0)! appelle la fonction de temporisation \verb!timer! au moins toutes les \verb!DeltaT! ms.
%\end{block}
%\pause
%\begin{block}{Fonction d'actualisation}
%\verb!glutPostRedisplay();! est la fonction qui demande à GLUT de calculer et d'afficher une nouvelle image sur l'écran.
%\end{block}
%\end{frame}


\begin{frame}
\frametitle{}
\begin{center}
{\Huge TP: ~\\
~\\
Simulation d'une flotte de robots
}
\end{center}
%\begin{figure}[h]
%\centering
%\includegraphics[width=0.8\textwidth]{img/end}
%\end{figure}
\end{frame}

\section{La librairie mathématique Eigen}

\begin{frame}
\frametitle{Commandes principales}
\begin{block}{Eigen::Vector3f v}
\begin{itemize}
\item Vecteur de $3$ floats x,y,z: Eigen::Vector3f(x,y,z)
\item Acces aux valeurs: v[i]
\item Compatible avec les opérations standards sur des vecteur: 3*u - 2*v est une syntaxe valide
\end{itemize}
\end{block}


\begin{block}{Produit scalaire $\langle u , v \rangle$}
\begin{itemize}
\item Calculé par u.dot(v)
\item Norme d'un vecteur: sqrt(u.dot(u)) ou u.norm()
\item Vecteur normalisé: $\textrm{u}_{\textrm{norm}}$ = u.normalized()
\item Autre version (inplace): u.normalize()
\end{itemize}
\end{block}

\begin{block}{Produit vectoriel $u \wedge v$}
\begin{itemize}
\item Calculé par u.cross(v)
\end{itemize}
\end{block}

\end{frame}


%\fi

\end{document}